\documentclass{article}
\usepackage{amsmath}
\usepackage{hyperref}  % Ajoute ce package pour gérer les URL

\begin{document}

\section{Modèle relationnel}

\subsection{Modèle relationnel}
Le modèle relationnel est une manière de modéliser les relations existantes entre plusieurs informations, et de les ordonner entre elles.  
Cette modélisation qui repose sur des principes mathématiques mis en avant par Edgar Franck Codd est souvent retranscrite physiquement dans une base de données.

\subsubsection{Objet}
Un objet est représenté par un ensemble de caractéristiques qui le rendent unique. On nomme ces tuples, des entités.

\subsubsection{Relation}
Une relation est un ensemble d’attributs et d’entités.  
(Dans le vocabulaire des bases de données, on utilise indifféremment les termes relation et table.)

\subsubsection{Contrainte d’intégrité}
Une contrainte d’intégrité est une propriété vérifiée à tout instant et qui garantit la cohérence des données.  
(On appelle clé primaire l’attribut qui garantit l’unicité de l’entité.)

\subsubsection{Contrainte d’entité}
La contrainte d’entité garantit que chaque entité d’une relation est unique et l’identifie de manière non ambiguë.

\subsubsection{Contrainte de référence}
Dans le MCD, une association entre deux relations est (souvent) traduite par une clé étrangère. La contrainte de référence garantit qu’une entité d’une relation B mentionne une entité existante dans une relation A.

\subsubsection{Schéma relationnel}
L’ensemble des relations constitue un schéma relationnel.

\paragraph{Représenter une entité}
\[
\texttt{Colis(numéro entier, masse entier, longueur entier, largeur entier, hauteur entier, statut caractères)}
\]

\subsubsection{MCD}
Le MCD permet de représenter le système d'information indépendamment de son aspect informatique.

\subsubsection{MLD}
\url{https://louisvandevelde.be/index.php?dos=my&fic=meris}

\section{POO}

\subsection{Instance}
Une instance est une construction concrète, produite à partir de l’objet modèle. L’instance est manipulable dans le programme.

\subsection{Classe}
Une classe est une modélisation d’un élément du monde réel.  
Une classe possède :
\begin{itemize}
  \item des attributs : il s’agit des caractéristiques spécifiques de l’objet (données),
  \item des méthodes : il s’agit des fonctionnalités que peut réaliser l’objet (traitements et services).
\end{itemize}

\subsection{Récursivité}

\subsubsection{Fonction récursive}
Une fonction récursive est une fonction :
\begin{itemize}
  \item qui s’appelle elle-même tant qu’une condition est vérifiée,
  \item possède une condition d’arrêt, pour ne pas s’appeler indéfiniment.
\end{itemize}

\section{Linux}

\subsection{Shell}
Le shell est un programme qui sert d’interface entre le noyau et l’utilisateur.

\subsection{Liste chaînée}
Dans une liste chaînée, chaque élément :
\begin{itemize}
  \item prend une place libre quelconque en mémoire,
  \item connaît l’emplacement de l’élément suivant.
\end{itemize}

\subsection{Pile}
Une pile est une liste chaînée dans laquelle on n’accède qu’au dernier élément.  
Les piles (stack) sont fondées sur le principe du dernier arrivé premier sorti : \textbf{Last In First Out}.

\subsection{File}
Une file est une liste chaînée dans laquelle on accède au premier et au dernier élément.  
Une file est fondée sur le principe : \textbf{First In First Out}.

\section{Arbres}

\subsection{Arbre}
Un arbre est défini par :
\begin{itemize}
  \item un nœud particulier qui constitue la racine,
  \item plusieurs sous-ensembles d’autres arborescences reliées à la racine.
\end{itemize}

On nomme :
\begin{itemize}
  \item nœud-fils l’ensemble des nœuds reliés à un même nœud-père,
  \item feuilles les nœuds qui n’ont pas de fils.
\end{itemize}

\subsection{Taille}
La taille d’un arbre est le nombre de nœuds de la structure.

\subsection{Hauteur / profondeur}
La hauteur (ou profondeur) d’un arbre est la longueur du plus grand chemin entre la racine et une feuille.

\section{Algorithmie}

\subsection{Coût / Complexité temporelle}
Le coût (ou complexité temporelle) représente le nombre d’étapes que l’algorithme doit réaliser pour exécuter sa tâche.

\subsection{Terminaison}
La terminaison d’un programme est la vérification qu’il finit par s’arrêter, c’est-à-dire que le nombre d’instructions exécutées est fini.  
Pour prouver la terminaison dans une boucle, on cherche un variant de boucle : une expression qui change à chaque itération, jusqu’à un cas limite.

\subsection{Variant de boucle}
Un variant de boucle est une expression numérique qui doit être modifiée à chaque itération de la boucle et qui doit garantir que la boucle finira par se terminer.

\subsection{Correction d'un programme}
La correction d’un programme est le fait de vérifier qu’il réalise effectivement ce qui était prévu.  
Dans une boucle, on cherche un invariant de boucle : une expression qui est vraie avant chaque itération.

\subsection{Invariant de boucle}
En programmation, une boucle est une instruction qui permet de répéter l'exécution d'une partie d'un programme. Un invariant de boucle est une propriété qui est vraie avant et après chaque répétition.

\section{Graphes non orientés}

\subsection{Définition}
Un graphe est une collection d’éléments mis en relation entre eux.

\subsection{Ordre}
L’ordre du graphe est le nombre de ses sommets.

\subsection{Non orienté}
Un graphe est non orienté quand ses arêtes peuvent être parcourues dans les deux sens.

\subsection{Adjacence}
Deux sommets reliés par une arête sont adjacents.

\subsection{Degré d’un sommet}
Le degré d’un sommet est le nombre d’arêtes de ce sommet.

\subsection{Complétude}
Un graphe est complet si tous les sommets sont adjacents à tous les autres.

\subsection{Propriétés}
La somme des degrés d’un graphe est paire.

\[
\sum_{s \in S} \text{deg}(s) = 2A
\]

\subsection{Représentation en mémoire - Sommets adjacents}
Dans un graphe orienté, les sommets adjacents n’ont pas la même position : ils peuvent être des prédécesseurs ou des successeurs.

\subsection{Matrice d’adjacence}
Dans une matrice d’adjacence, le sommet représenté sur la ligne est le départ de l’arête (le prédécesseur).  
Il est aussi possible d'établir une matrice d'adjacence pour un graphe orienté. Le principe reste le même : si le sommet $i$ (ligne) est lié au sommet $j$ (colonne), nous avons un 1 à l'intersection (0 dans le cas contraire).  
En ligne nous avons les prédécesseurs et en colonne les successeurs.

\subsection{Dictionnaire d’adjacence}
Un dictionnaire d’adjacence liste les sommets adjacents à chaque sommet. On peut faire un dictionnaire des prédécesseurs et un des successeurs.

\end{document}
